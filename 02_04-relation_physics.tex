\section{Relation with physics}

Let $P$ be a principal $G$-bundle and assume for simplicity that $G$ is a matrix Lie group, like $SU(n).$ Let $\rho: G
\to GL(V)$ be a finite-dimensional representation of $G.$ Construct the associated bundle $P \times_{G} V.$ Then a
local section of $P$ will induce a local trivialisation of $P \times_{G} V.$

Let $s$ be a local sections of the associated bundle $P \times_{G} V.$ On some local trivialisation $(U, \sig)$ of $P,$
induced by a local section $\sig:U \to P,$ the section $s$ can be considered as a $V$-valued function. This is the
point of view most physicists take.

If we consider $s$ as a $G$-equivariant function $\bar{s}:P \to V,$ then on $U \subset M$ the section $s$ is realized
as a $V$-valued function as $\bar{s}\comp \sig.$ In these local coordinates, the covariant derivative $s \mapsto
(ds)^{H}$ takes the form
\begin{align*}
 (ds)^{H}\comp \sig &= \sig^{*}(ds + \om \dwed s) \\
&=d\sig^{*}s + \sig^{*}\om \dwed \sig^{*}s \\
&= (d+\om_{U})\sig^{*}s.
\end{align*}
That is, on local trivialisation, a connection $\om$ takes the form $d+\om_{U}$, where $\om_{U}$ transforms according
to the rule (\ref{JB:1}). This is the usual form of a connection one encounters in physics.

The statement that two connections differ by an element of $\dfp{1}{\mfk{g}}$ translates, on the level of vector
bundles, to the one that two connection $\nab, \prm{\nab}$ differ by an element of $\Gamma(M, T^{*}M\otimes E).$ The
curvature $F_{\om}=d\om + \frac{1}{2} \om \wed \om$ lies in $\dfp{2}{׿\mfk{g}}$ and therefore corresponds to an
element $F \in \Gamma(T^{*}M \otimes T^{*}M \otimes E),$ which is also known as the curvature connection $\nab$ on $E.$

The local form on $M$ of the curvature $F_{\om}=d\om + \frac{1}{2}\lie{\om}{\om}$ is $F_{u}=d \om_{U} +
\frac{1}{2}\lie{\om_{U}}{\om_{U}},$ the usual field strength tensor in field theory.