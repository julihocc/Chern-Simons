\section{Principal Fiber Bundles}

Classical gauge theories are mathematically described using principal fiber bundles.

\begin{defn}[\cite{DB}, 1.1.1]
 \label{JB:2.1}
 A principal fiber bundle $\pi: P \to M$ with structure group $G$ consists of smooth manifolds $P, M$ and a Lie Group
$G$ together with a smooth surjective projection map $\pi: P \to M,$ where the Lie group $G$ has a free smooth right
action on $P$ and $\pi^{-1}(\pi(p))=\set{pg : g \in G}.$ If $x \in M,$ then $\pi^{-1}(x)$ is called the fiber above $x.$

Furthermore, we require that for each $x \in M$ there exists an open set $U$ containing $x$ and a diffeomorphism
$T_{U}: \pi^{-1}(U) \to U \times G$ of the form $\phi_{U}(p)=(\pi(p), s_{U}(p)),$ where $s_{U}: P \to G$ has the
property $s_{U}(pg)=s_{U}(p)g$ for all $g \in G, p \in \pi^{-1}(U).$
The map $T_{U}$ is called local trivialization (or, in physics language, a choice of gauge).
\end{defn}

Let $T_{U}: \fib{U} \to U\times G$ and $T_{V}:\fib{V} \to V \times G$ be two local trivializations of a principal
fiber bundle $\pi : P \to M.$ The \emph{transition function} from $U$ to $V$ is the map $g_{UV}: U \cap V \to G$
defined, for $x = \fib{p} \in U \cap V,$ by $g_{UV}(x)=s_{V}(p)s_{U}(p)^{-1}.$ Thus is independent of the choice of $p
\in \fib{x}.$

\begin{thm}[\cite{DB}, 1.1.5]
 \label{JB:2.2}
A principal bundle $\pi: P \to M,$ with structure group $G$ is a trivial bundle if and
only if it admits a global section $M \to P.$ 
\end{thm}


\begin{proof}
 Let $ \sig: U \to P$ be a given local section. Then define the smooth map $T_{U}: \fib{U} \to U \times G$ by
$T_{U}(\sig(x)g)=(\pi(p), g).$ We verify that indeed $T_{U}(\sig(x)gh)=(\pi(p), gh)$ so that $T_{U}$ is a local
trivialization. Conversely, let $T_{U}: \fib(U) \to U \times G$ be a given local trivialization, then define a section
$\sig : U \to \fib{U}$ by $\sig(x)=\phi_{u}^{-1}(x,e).$ This, we see that local sections correspond to a local
trivializations. In particular, this implies that a global sections exists if ans only if the bundle is trivial.
\end{proof}

\begin{exmp}[Square root]
 \label{JB:2.3b}
The map $z \mapsto z^{2}$ in $S^{1}$ induces a principal bundle $\pi : S^{1} \to S^{1}$ with structure group $\Z^{2}.$
It is locally trivial since, locally, on the circle there always exists a smooth square root function. Since the total
space is $S^{1}$ and not $S^{1}\times \Z_{2}$ this bundle is not trivial. Hence there does not exist a continuous
square root function on $S^{1}.$
\end{exmp}


\begin{exmp}[Hopf fibration]
 \label{JB:2.3a}
Identify $\R^{3}$ with $\C \times \R$ by $(x_{1}, x_{2}, x_{3}) \mapsto (z=x_{1} + ix_{2}, x=_{x_{3}})$ and $\R^{4}$
with $\C^{2}$ by identifying $(x_{1},...,x_{4}) \mapsto (z_{1}=x_{1}+ix_{2}, z_{2}=x_{3}+ix_{4}).$ Then the unit sphere
$S^{2}$ in $\R^{3}$ is identified with $\set{(z,x)| \abs{z}^{2}+\abs{x}^{2}}$ and $S^{3}$ is identified with
$\set{(z_{1}, z_{2}| \abs{z_{1}}^{2}+\abs{z_{2}}^{2})=1}.$

The Hopf fibration is defined by
$$
p(z_{1}, p_{2})=(2z_{1}z_{2}^{*}, \abs{z_{1}}^{2}\abs{z_{2}}^{2}).
$$

Then $p$ maps $S^{3}$ onto $S^{2}$ as can be checked: $$4\abs{z_{1}}^{2}\abs{z_{2}}^{2}+
(\abs{z_{1}}^{2}-\abs{z_{2}}^{2})^{2}=(\abs{z_{1}}^{2}+\abs{z_{1}}^{2})^{2}=1.
$$ It can be shown that $p$ maps elements of $S^{3}$ to the same point in $S^{2}$ if and only if these points are the
same up to a factor $\lam \in U(1).$ The bundle is not global trivial, but for Hopf fibration it is enough to remove a
single point $m \in S^{2},$ thus one can take $U=S^{2}-{m}$ as trivializing neighbourhoods, and any point in $S^{2}$
has a neighbourhood of this form. Hence $p: S^{3} \to S^{2}$ is a principal $U(1)-$ bundle.
\end{exmp}

