\section{Gauge transformation}

Equations (\ref{SA:1.21}) and (\ref{SA:1.23}) show that $A$ determines $B,$ as well as part of $E.$ Notice that $B$ eis left invariant by the transformation
\begin{equation}
\label{SA:1.40}
A \mapsto A'=A + \nab \chi,
\end{equation}
for any scalar function $\chi.$ The invariance of $E$ es accomplished by the transformation 
\begin{equation}
\label{SA:1.41}
\Phi \mapsto \Phi'=\Phi- \pt{\chi}.
\end{equation}

The transformations (\ref{SA:1.40}) and (\ref{SA:1.41}) are called \emph{gauge transformations,} and the invariance of the fields under such transformations is called \emph{gauge invariance.}

In the lenguage of covariance, we see that $A^{\alpha}=(\Phi, A)$ is not unique: the same electromagnetic field tensor $F^{\alpha \beta}$ can be obtained from potential
\begin{equation}
\label{SA:1.42}
A^{\alpha}=\var{\Phi- \pt{\chi}, A + \nab \chi}.
\end{equation}

Substituting (\ref{SA:1.42}) into (\ref{SA:1.28}) we obtain
\begin{align*}
F^{\alpha \beta} &= \pu{\alpha}A^{\beta} + \pu{\alpha}\pu{\beta}\chi - \pu{\beta}A^{\alpha} - \pu{\beta}\pu{\alpha} \chi \\
&= \pu{\alpha}A^{\beta} - \pu{\beta}A^{\alpha} + \comm{\pu{\alpha}}{\pu{\beta}} \chi \\
&= \pu{\alpha}A^{\beta} - \pu{\beta}A^{\alpha},
\end{align*}
using the fact that $\comm{\pu{\alpha}}{\pu{\beta}}=0.$
\begin{rem}
The transformation $A^{\alpha} \mapsto {A'}^{\alpha}= A^{\alpha} + \pu{\alpha} \chi$ is a gauge transformation.
\end{rem}