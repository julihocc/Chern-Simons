\section{Lie groups and Lie algebras}

\begin{defn}
 \label{DB:0.3.1}
Let $G$ be an $n$-manifold and a group such that the groups operation  $G \times G \to G$ given by $(g_{1},g_{2})
\mapsto g_{1}g_{2}$ and the function $G \to G$ given by $g \mapsto g^{-1}$ are $C^{\infty}$ maps. Then $G$ is called a
\emph{Lie group.}
\end{defn}

\begin{defn}
 \label{DB:0.3.2}
Let $L_{g}:G \to G$ be defined by $L_{g}(\prm{g})=g\prm{g};$ $L_{g}$ is a diffeomorphism. Let $e$ be the identity
element of $G,$ and let $A \in T_{e}G.$ Define $\bar{A}\in \Gamma(TG)$ by $\bar{A}_{g}=L_{g*}(A);$ $\bar{A}$ is called
the \emph{left-invariant vector field} determined by A.
\end{defn}

\begin{defn}
 \label{DB:0.3.3}
Let $\la{G}=T_{e}G$ and ,for $A,B \in \la{G}$, define $[A,B] \in \la{G}$ by
$\lie{A}{B}=\lie{\bar{A}}{\bar{B}}_{e}.$ Note that it is \emph{anti-symmetric} and it satisfies the \emph{Jacobi
identity}. Then $\la{G},$ together with the bracket operation $\lie{}{×},$ is called the \emph{Lie algebra} of $G.$
\end{defn}


 For $A \in \la{G},$ we can prove that $\bar{A}$ is a complete vector field. Let $\set{\vphi_{t}}$ be the
one-parameter
group of diffeomorphism generated by $\bar{A} \in \la{G}.$ Let $\gam: \R \to G$ be the curve through $e$ defined by
$\gam(t)=\vphi_{t}(e).$

We prove that $\gam(s+t)=\gam(s)\gam(t)$ (group multiplication). Let $s \in \R$ be fixed and let
$\gam_{1}(t)=\gam(s+t),$ while $\gam_{2}(t)=\gam(s)\gam(t).$ Then
$\prm{\gam}_{1}(t)=\prm{\gam}(s+t)=\bar{A}_{\gam(s+t)}$ and
\begin{align*}
 \prm{\gam}_{2}(t)&=L_{\gam(s)}{*}(\prm{\gam}(t)) \\
&= L_{\gam(s)}{*}(\bar{A}_{\gam(t)}) \\
&= L_{\gam(s)}*(L_{\gam(t)}*A)=\bar{A}_{\gam(s)\gam(t)}.
\end{align*}
Thus $\gam:\R \to G$ is homomorphism. Conversely, given a curve and homomorphism $\sig:\R \to G,$ then $\psi_{t}:G \to
G,$ defined by $\psi_{t}(g)=g\sig(t)$ is a one-parameter group of diffeomorphism of $G$ such that
$$
\bar{B}_{g}\equiv \frac{d}{dt}\psi_{t}(g)|_{t=0}
$$
defines the left-invariant vector field $\bar{B}$ determined by $B\equiv \bar{B}_{e}.$ Thus, there is a one-to-one
correspondence $A \leftrightarrow \gam.$

\begin{defn}
\label{DB:0.3.4}
 We define the \emph{exponential map} $\exp:\la{G} \to G$ by $\exp(A)=\gam(1).$ Note that $\gam(t)=\exp(tA),$ and
$\gam_{t}(g)=g\gam(t)=g \exp(tA).$
\end{defn}

\begin{exmp}
\label{DB:0.3.5}
 Let $V$ be a vector space with $\dim V = m <\infty,$ and let $GL(V)$ be the group of invertible linear functions $F:V
\to V.$ By regarding $GL(V)$ as a group of matrices, it is simple to see that $GL(V),$ which is a open subset of
$\R^{m^{2}},$ is a Lie group.

Let $I \in GL(V)$ be the identity, and denote $T_{I}(GL(V))$ by $\la{Gl}(V).$ Note that $\la{Gl}(V)$ can be
identified whit the vector space of \emph{all} linear functions $A:V \to V,$ the correspondence being
$$
A \leftrightarrow \frac{d}{dt}(I+tA)\bigg|_{t=0}.
$$

For $A \in \la{Gl}(V),$ let
$$
\Exp(A)= I + A + \frac{1}{2!}A^{2} + ...
$$
It can be proved that the sum converges, and that $$\Exp((t+s)A)=\Exp(tA)\Exp(sA).$$ Thus, $\Exp(A)\Exp(-A)=I$ and so
$\Exp(A)\in \la{Gl}(V).$ Note that $t \mapsto \Exp(tA)$ is a curve and a homomorphism with
$$
\frac{d}{dt}\Exp(tA)\bigg|_{t=0}=A.
$$
It follows from the discussion previous to definition \ref{DB:0.3.4} that $\Exp$ is the exponential map for $GL(V).$ In
\ref{DB:0.3.10}, we will prove that, for $A,B \in \la{Gl}(V),$ $\lie{A}{B}=AB-BA.$
\end{exmp}

\begin{defn}
 \label{DB:0.3.6}
A \emph{Lie subgroup} of a Lie group $G$ is a submanifold (of $G$) that is also a subgroup. A Lie subgroup $H$ of $G$
is itself a Lie group. Since the homomorphisms $\gam:\R \to H$ are also homomorphisms into $G,$ we have that $\exp:
\la{H} \to H$ is just $\exp: \la{G} \to G$ restricted to $\la{H}.$
\end{defn}

\begin{thm}
 \label{DB:0.3.7}
Let $G$ and $\prm{G}$ be Lie groups, and let $F:G \to \prm{G}$ be a $C^{\infty}$ homomorphism. The $F_{*e}: \la{G} \to
\prm{\la{G}}$ is a linear function such that $F_{*e}(\lie{A}{B})=\lie{F_{*e}A}{F_{*e}},$ i.e., $F_{*e}$ is a
homomorphism of Lie algebras.
\end{thm}

\begin{proof}
 Note that $$F \comp L_{g}(\prm{g})=F(g\prm{g})=F(g)F(\prm{g})=(L_{F(g)}\comp F)(\prm{g}).$$ Thus
$$
F_{*g}(\bar{A}_{g})=F_{*g}(L_{*g}A)=L_{F(g)*\prm{e}}(F_{*e}A)=(\overline{F_{*e}A})_{F(g)},$$ and so
$F_{*}(\bar{A})=(\overline{F_{*e}}A).$ Thus
$$
F_{*e}(\lie{A}{B})=\lie{F_{*}(\bar{A})}{F_{*}(\bar{B})}_{\prm{e}}=\lie{(\overline{F_{*e}A})}{(\overline{F_{*e}B})}_{\prm
{e}}=\lie{F_{*e}A}{F_{*e}B}.
$$
\end{proof}

\begin{defn}
 \label{DB:0.3.8}
For $g \in G,$ let $\op{Ad}_{g}:G \to G$ be the $C^{\infty}$\emph{adjoint isomorphism} given by $\op{Ad}_{g}(\prm{g})=g
\prm{g}g^{-1}.$ We let $\la{Ad}_{g}: \la{G} \to \la{G}$ be the induced isomorphism of $\la{G}$ provided by theorem
\ref{DB:0.3.7}, i.e., $\la{Ad}_{g}=\la{Ad}_{d*e}.$ Let $\Lad{}: G \to GL(\la{G})$ be the homomorphism $g\mapsto
\Lad{g}.$ Then theorem \ref{DB:0.3.7} gives us an induced homomorphism $\lad{}: \alg{G} \to \alg{Gl}(\alg{G}),$ i.e.,
$\lad{}=\Lad{*e}.$
\end{defn}

\begin{thm}
\label{DB:0.3.9}
 For $A,B \in \alg{G},$ we have
$$
\lad{}(A)(B)=\frac{\p^{2}}{\p s \p t}\var{\exp(tA)\exp(sB)\exp(-tA)}\bigg|_{s,t=0}=\lie{A}{B}.
$$
\end{thm}
\begin{proof}
 Let $\set{\vphi_{t}}$ be the one-parameter group generated by $\bar{A}.$ By the end of definition \ref{DB:0.3.4}, we
have $\vphi_{t}(g)=g \exp tA.$ Using $L_{\bar{A}}\bar{B}=\lie{\bar{A}}{\bar{B}},$ we have (at $s=t=0$)
\begin{align*}
 \lie{A}{B}&=\lie{\bar{A}}{\bar{B}}_{e}=\dd{t} \vphi_{-t*}\var{\bar{B}_{\vphi_{t}(e)}} \\
&= \dd{t} \vphi_{-t*} \var{ \dd{t} \vphi_{t}(e)\exp(sB)} \\
&=\dd{t}\dd{s}\vphi_{-t}\var{\vphi_{t}(e)\exp(sB)} \\
&= \frac{\p^{2}}{\p t \p s}\var{\exp (tA)\exp(sB)\exp(-tA)} \\
&= \dd{t}\Lad{}\var{\exp(tA)}(B)=\Lad{*e}(A)(B) \\
&= \lad{}(A)(B)
\end{align*}
\end{proof}

\begin{cor}
 \label{DB:0.3.10}
If $G$ is any Lie subgroup of $GL(V),$ then the bracket operation on $\alg{G} \subset \alg{Gl}(V)$ is given by
$\lie{A}{B}=\liex{A}{B}.$
\end{cor}
\begin{proof}
 By \ref{DB:0.3.6} it suffices to consider the case in which $G=GL(V).$ Using theorem \ref{DB:0.3.9} with $\exp=\Exp$
(see example \ref{DB:0.3.5}), we have
$$
\lie{A}{B}=\frac{\p^{2}}{\p s \p t}(\Exp(tA)\Exp(sB)\Exp(-tA))\bigg|_{s,t=0}=\liex{A}{B}.
$$
\end{proof}

\begin{defn}
 \label{DB:0.3.11}
Let $e_{1},...,e_{n}$ be a basis for the Lie algebra $\alg{G}$ of $G.$ The \emph{structure constants} $\ct{i}{j}{k}\in
\R$ are defined by $\lie{e_{i}}{e_{j}}=\sum \ct{i}{j}{k} e_{k.}$ Note that $\lie{e_{i}}{e_{j}}=-\lie{e_{j}}{e_{i}},$
which implies $\ct{i}{j}{k}=-\ct{i}{j}{k}.$ Similarly, the Jacobi identity implies
$$
\sum_{m}\ct{i}{m}{h}\ct{j}{k}{m} + \ct{k}{m}{h}\ct{i}{j}{m} + \ct{j}{m}{h}\ct{k}{i}{m}=0,
$$
for all $h,i,j,k.$
\end{defn}

\begin{exmp}[$SU(n)$, the Special Unitary Group]
The computation of the Lie algebra of a Lie group of matrices is illustrated here for the group $SU(n),$ which is
frequently used in elementary particle physics.

Let $\alg{Gl}(n, \C)$ be the space of all $n \times n$ matrices with complex entries. For $A \in \alg{Gl}(n, \C),$ let
$A^{*}$ denote the conjugate of the transpose of $A.$ Recall that the unitary group is $$U(n)=\set{A \in \alg{Gl}(n,
\C)| A A^{*}=I}$$ and $$SU(n)=\set{A \in U(n)| \det A =1}.$$

If $t \mapsto A(t)$ is a curve in $U(n)$ with $A(0)=I,$ then at $t=0,$ we have
\begin{align*}
 0&=\dd{t}(I)=\dd{t}\set{A(t)A(t)^{*}} \\
&= \prm{A}(0)A(0)^{*}+ A(0)\prm{A}(0)^{*}=\prm{A}(0)+\prm{A}(0)^{*}.
\end{align*}

Thus, for $\alg{S}=\set{B \in \alg{Gl}(n, \C)| B + B^{*}=0},$ we have $\alg{S} \supset \alg{U}(n),$ the Lie algebra of
$U(n).$

Conversely, if $B \in \alg{S},$ then $(\Exp B)(\Exp B)^{*}=(\Exp B)(\Exp B^{*})=I,$ and so $\Exp B \in U(n).$ At $t=0,$
$$
B=\dd{t}\Exp tB \in \alg{U}(n),
$$
whence $\alg{U}(n)= \alg{S}.$

The Lie subalgebra $\alg{SU}(n)$ of $SU(n)$ is the subalgebra of $\alg{U}(n)$ consisting of matrices with trace $0,$
i.e., $$
\alg{SU}(n)0\set{B \in \alg{U}(n)| \alg{tr} B=0}.
$$
This follows fro the formula $\det (\Exp B)=e^{\op{tr}B},$ which is valid for any $n \times n$ matrix. We can prove
this formula as follows, Let $f(t)=\det (\Exp tB). $ at $h=0,$ we have
\begin{align*}
 \prm{f}&=\dd{t}f(t+h)=\dd{t}\left[\det (\Exp tB) \det(\Exp hB)\right] \\
&=\det(\Exp tB)\dd{h}\det(I+hB)\\
&=\det(\Exp tB)\alg{tr} B =(\alg{tr}B)f(t).
\end{align*}
Thus, $f(t)=f(0)e^{(\op{tr}B)t}=e^{(\op{tr}B)t},$ and setting $t=1$ yields the result.


\end{exmp}








