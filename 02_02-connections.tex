\section{Connections and gauge potential}

\begin{defn}[\cite{DB}, 1.2.1]
\label{JB:3.1}
A \emph{connection} assigns to each $p \in P$ a subspace $H_{p} \subset T_{p}P$ such that
\begin{enumerate}[(i)]
 \item $T_{p}P= V_{p} \bigoplus H_{p},$ where $V_{p}=\set{X \in T_{p}P| \pi_{*}(X)=0}.$
\item $R_{g*}H_{p}=H_{p},$ where $R_{g}$ is multiplication by the right
\item For each $p\in P,$ there exist a neighbourhood $U$ and vector fields $X_{1},...,X_{n}$ on $U$ such that $H_{p}$
is spanned by $X_{1}(p),...,X_{n}(p)$ for all $p \in U.$
\end{enumerate}
\end{defn}

We will also use the following equivalent definition:
\begin{defn}[\cite{DB}, 1.2.2]
 \label{JB:3.2}
A \emph{connection} is a $\mfk{g}-$valued 1-form $\om$ on $P$ satisfying thw following conditions:
\begin{enumerate}[(i)]
 \item For the fundamental vector fields $A^{*}(p):= \frac{d}{dt}(p \exp(tA))|_{t=0},$
$$
\forall p \in P, \om_{p}(A^{*})=A,
$$
\item For any $g \in G,$
$$
R^{*}_{g}\om =\op{ad}_{g^{-1}}\om,
$$
where $\op{ad}_{g^{-1}} \in Aut(\mfk{g})$ is the adjoint action of $g$ on $\mfk{g}.$ To be more precise, we require
$(R_{g*}\om)(p)=\op{ad}_{g^{-1}}\om_{p},$ i. e., $\om_{pg}(R_{g*}X_{p})=ad_{g^{-1}}(\om_{p}(X_{p})),$ where $ad:G \to
GL(\mfk{g})$ denotes the adjoint action of $G$ on $\mfk{g}.$
\end{enumerate}
\end{defn}

The equivalence of definitions \ref{JB:3.1} and  \ref{JB:3.2} can be seen as follows: given a connection in the form of
definition \ref{JB:3.1}, one defines a $\mfk{g}-$valued 1-form $\om$ by $\om(A^{*})=A$ and $\om_{p}(X_{p})=0$ for all
$X_{p} \in H_{p}.$

Conversely, given a connection 1-form $\om$ as in definition \ref{JB:3.2} define $H_{p}=\set{X \in T_{p}P |
\om_{p}(X_{p})=0}.$ Since the action of $G$ is free, the map $A \mapsto A^{*}_{p}$ is injective. 

In order to see this, let $A$ be such that $A^{*}_{p}=0,$ then 
\begin{align*}
 \frac{d}{dt}(p \exp(tA))&= \frac{d}{ds}(p \exp((s+t)A))|_{s=0} \\
&= \frac{d}{ds}(p\exp(sA)\exp(tA))|_{s=0} \\
&= R_{\exp(tA)*}\frac{d}{ds}(p \exp(sA))|_{s=0} \\
&= R_{\exp(tA)*} A^{*}_{p}=0.
\end{align*}

So one obtains a vector space isomorphism $\mfk{g}=V_{p}.$ From part 1) of definition \ref{JB:3.2}, it follows that
$H_{p} \bigoplus V_{p}=T_{p}P.$

It can been shown that $[A^{*},B^{*}]_{p}=[A,B]^{*}_{p}$ for all $A,B \in \mfk{g},$ where $[\cdot, \cdot]$ on vector
fields is the usual Lie-bracket, i.e., $[X,Y]=\frac{d}{dt}(\phi^{-1}_{t*}Y_{\phi_{t}(p)})|_{t=0}$ where
$\frac{d}{dt}\phi_{t}=X$ in a neighbourhood of $p.$

\begin{prop}
\label{JB:3.4}
The vector fields on M can be identified with the $G-$invariant horizontal fields on $P.$ 
\end{prop}
\begin{proof}
 The map $\pi_{*}$ establish an isomorphism between $H_{p}$ and $T_{\pi{p}}M,$ so it indeeds identifies a subspace of
$T_{p}P$ with the tangent space of M. This allows us to lift vector fields on M to unique horizontal vector fields on
$P.$ This horizontal lift is $G-$invariant, i.e., $R_{g*}\tilde{X}=\tilde{X},$ since $R_{g*}H_{p}=H_{pg}.$ Conversely,
every $G-$invariant vector field $\tilde{X}$ is a lift of a vector field $X=\pi_{*}\tilde{X}$ on $M,$ which is
well-defined since $\tilde{X}$ is $G-$invariant and $\pi$ is surjective. 
\end{proof}

\begin{lem}
 For a horizontal lift $\tilde{X}$ of $X,$ we have $[A^{*}, \tilde{X}]=0,$ for all $A \in \mfk{g}.$
\end{lem}

\begin{proof}
 Use the formula $[A^{*}, \tilde{X}]= \frac{d}{dt}\var{\phi^{-1}_{t*}\tilde{X}_{\phi_{t}(p)}}|_{t=0},$ where
$\phi_{t}(p)= p \exp (tA),$ and the $G-$invariance of $\tilde{X},$ which implies $\phi_{t*}\tilde{X}=\tilde{X},$ for
all $t$.
\end{proof}

\begin{rem}
 By choosing a local trivialisation, i.e, a local section $\sig : U \to P,$ one can pull back $\om$ to a 1-form
$\om_{U}= \sig^{*}\om$ on $U \subset M.$ In the particular case that $G$ is a matrix Lie group and two local sections
$\sig:U \to P$ and $\sig: V \to P$ are given such that $U \cap V \neq \emptyset,$ then the transformation rule
between $\om_{U}$ and $\om_{V}$ is given by
\begin{equation}
 \label{JB:1}
\om_{V}=g_{uv}^{-1} d g_{UV} + g^{-1}_{UV} \om_{U} g_{UV}.
\end{equation}
(C.f. [\cite{DB}], definition 1.2.3 and theorem 1.2.5.)

Conversely, a collection of local $\mfk{g}-$valued 1-forms $\set{\om_{U}},$ where $\set{U_{i}}$ is a cover of $M$ such
that $P$ is locally trivial on $U_{i},$ for all $i,$ satisfying the transformation rule (\ref{JB:1}) glue together into
a global $\mfk{g}-$valued 1-form $\om$ on $P$ satisfying the conditions of definition \ref{JB:3.2}.

The pullback $\om_{U}$ on $M$ are known as gauge potentials in physics. The transformation property (\ref{JB:1}) is
what physicists may recognize as the transformation rule of a \emph{gauge potential} in gauge theories. A connection on
$P$ is therefore also known as a gauge potential. The choice of the local sections is called a choice of gauge.
\end{rem}

 \begin{defn}
  \label{JB:3.7}
A gauge theory on $M$ with a gauge group $G$ consists of a principal $G-$bundle $\pi:P \to M$ endowed with a connection
$\om.$
 \end{defn}




