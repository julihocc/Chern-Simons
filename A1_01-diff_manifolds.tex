\section*{Differential manifolds}

\begin{defn}[Differential manifold]
\label{marathe_defn-1.1}
 Let $M$ be a topological space and $F$ a Banach space. 

\begin{itemize}
 \item A \tb{(local coordinate) chart} $(U, \phi)$ is a open subset $U \subset M$ and a homeomorphism $\phi: U \to \phi(U) \subset F.$
\item If $M$ admits a family of charts such that covers $M$, $M$ is called a \tb{topological manifold}, $\mc{A}$ is said to be an \tb{atlas} for $M$. 
\item The homeomorphisms 
$$\phi_{ij}:= \phi_{i}\comp \phi_{j}^{-1}: \phi_{j}(U_{ij}) \to \phi_{i}(U_{ij}),$$ with $U_{ij}= U_{i} \cap U_{j}$ are called \tb{transition functions} of the atlas $\mc{A}.$
\item If $\set{\phi_{ij}}$ are  $C^{p}-$ diffeomorphisms, $0 < p \leq +\infty,$ $\mc{A}$ is called a \tb{differentiable atlas of class $C^{p}$.} 
\item The maximal differentiable atlas of class $C^{p}$ containing $\mc{A}$ es called \tb{th differentiable structure} on $M$ of class $C^{p}.$ If $p=+\infty$, $\mc{A}$ is also called \tb{smooth structure}. A manifold with a smooth structure is also called \tb{differentiable manifold.}
\item The dimension $dim M$ of $M$ is, by definition, the dimensi\'on of $F.$ If $F=\R^{n},$ $M$ is called a \tb{real manifold}, and if $F=\C^{n},$ $M$ is called a \tb{complex manifold}. 
\end{itemize}
  
\end{defn}

\begin{defn}[Differential application]
\label{marathe_defn-1.2}
 Let $M,N$ be a differentiable manifolds and $f:M \to N.$ We say that $f$ is \tb{differentiable} (or \tb{smooth}) if for each $(U, \phi) \in \mc{A}(M)$, $(V, \psi) \in \mc{A}(N)$, such that $f(U)\subset V,$ $\psi \comp f \comp \phi^{-1}$ is differentiable (or smooth). We shall denote the set of all smooth functions $:M \to N$ by $\smf{M,N}.$ When $N=\R$, we write $\smf{M}$ instead $\smf{M,\R}.$ 

A bijective differentiable $f\in\smf{M,N}$ is called a diffeomorphism \ti{if $f^{-1}$ is differentiable.} The set of all diffeomorphisms of $M$ with itself under compositions is a group denoted by $\diff{M}.$ 
 \end{defn}

\begin{defn}[Open submanifold]
   \label{marathe_defn-1.3}
Let $M$ be a differential manifold with differential structure $\mc{A}$ and let $U\subset M$ be open. The collection of all charts of $\mc{A}$ whose domain is a subset of $U$ is an atlas for $U,$ which makes $U$ into a differential manifold. This manifold is called an \tb{open submanifold} of $M$.
\end{defn}

Let $\R^{n}_{+}=\set{(x_{1}, \dots, x_{n})\in \R^{n}| x_{n} \geq 0}$ and we give it the realtive topology; the set $\R_{0}^{n}=\set{x\in \R^{n}| x_{n=0}}$ es called the boundary of $R^{n}_{+}.$ If $M$ is a topological space, a \tb{chart with boundary} for $M$ is a pair $(U, \phi)$ where $U$ is an open set of $M$ and $\phi:U\to \phi(U) \subset \R^{n}_{+}$ is a homeomorphism; so we have the notions of an \tb{atlas with boundary} and of an \tb{manifold with boundary}. The \ti{boundary} of $M$, denoted by $\p M$, is the subset of the points $x \in M$ such that there exists a chart with boundary $(U, \phi)$ with $x\in $ and $\phi(x)\in \R^{n}_{0}.$ The interior of $M$ is the set $\ti{Int}M = M \backslash \p M.$

\begin{rem}
 The differentiable structure of an $n-$manifold with boundary $M$ induces, in a natural way, a differentiable structure on $\p M$ and $\ti{Int} M$ of dimensions $n$ and $n-1$, respectively.
\end{rem}

Let $(U, \phi), (V, \psi)$ two charts of $M$ at $p\in M.$ The triples $(\phi, p, u), (\psi, p, v),$ for $u,v \in F,$ are said to be equivalent if 
$$
D(\psi \comp \phi^{-1})(\phi(p))\cdot u = v.
$$
A \tb{tanget vector} to $M$ at $p$ may be defined as an equivalence class $\encap{\phi, p, u}$ of such triples.

Alternatively, one can define a tanget vector to $M$ at $p$ to be an equivalence class of smooth curves on $M$ passing through $p$, where two curves are equivalents is given a chart, both compositions have the same ordinary derivative. 

The set or tangent vectors at $p$ is denoted by $T_{p}M$ and is a vector space isomorphic to $F$, called the \tb{tangent space} to $M$ at $p.$ The set
\[TM=\bigcup_{p\in M}T_{p}M\]
can be given the structure of a manifold. This manifold is called the \tb{tanget space} to $M$.