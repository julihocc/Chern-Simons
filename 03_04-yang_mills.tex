\section{Yang-Mills Fields}

The Yang-Mills fields form a special class of gauge fields that have been extensively investigated.

Let M be a connected manifold and let $P(M, G)$ be the principal bundle over $M$ with Lie group $G$ as the gauge group.
In this section we restrict ourselves to the space $\mc{A}(P)$ of gauge connections as our configuration space. If
$\om$ is a gauge connection on $P$ then, in a local gauge $t \in \Gamma(U, P)$ we have the corresponding gauge
potential
$$
A_{t}=t^{*}\om \in \Lam^{1}(U, \ad{}(P)).
$$

Locally, a gauge transformation $g \in \mc{G}$ reduces to a $G$-valued function $g_{t}$ on U, and its action on $A_{t}$
is given by
$$
g_{t}\cdot A_{t}=(ad g_{t}) \comp A_{t}+ g_{t}^{*}\Theta,
$$
where $\Theta$ is the canonical 1-form on $G$. It is customary to indicate this gauge transformation as follows
$$
A^{g}:=g \cdot A = g^{-1}Ag + g^{-1}dg
$$
when the section $t$ is understood.

The gauge field $F_{\om}$ is the unique 2-form on $M$, with values in the bundle $\ad{}(P)$, satisfying
$$
F_{\om}=s_{\Om},
$$
where $\Om$ is the curvature of the gauge connection $\om.$ In the local gauge $t$, we can write
$$
F_{\om}0 d^{\om}A_{s}= dA_{s}+\frac{1}{2}\lie{A_{s}}{A_{s}},
$$
where the bracket is taken as the bracket of bindle valued forms.

We note that it is always possible to introduce a Riemannian metric on a vector bundle over $M$. The manifold $M$ itself
admits a Riemannian metric. We assume that metrics are chosen on $M$ and the bundles over $M$ and the norm is defined
on the sections of these bundles as an $L^{2}$-norm is $M$ is not compact. However, to simplify the mathematical
considerations, we assume in the rest of this chapter that $M$ us a compact, connected, oriented, Riemannian manifold
and that $G$ is a compact, semisimple Lie group, unless otherwise indicated.

For a given $M$ and $G$, the principal $G$-bundles over $M$ are classified by $[M; BG]$, the set of homotopy classes of
maps of $M$ into the classifying space $BG$ for $G$. For $f \in [M; BG]$, let $P_{f}$ be a representative of the
isomorphism class of potentials on $P_{f}.$ Then, the Yang-Mills configuration space $\mc{A}_{M}$ is the disjoint union
of the spaces $\mc{A}(P_{f})$, i.e.
$$
\mc{A}_{M}= \cup\set{\mc{A}P_{f}| f \in [M; BG]}.
$$

The isomorphism class $[P_{f}]$ is uniquely determined by the characteristic classes of the bundle $P_{f}$. For
example, if $\dim M=4$ and $G=SU(2)$, then $c_{1}(P_{f})=0$ and hence $[P_{f}]$ is determined by the second Chern class
$c_{2}(P_{f})=c_{2}(V_{f})$, where $V_{f}$ is the complex vector bundle of rank two associated to $P_{f}$ by the
defining representation of $SU(2)$. Now the class $c_{2}(P_{f})$, evaluated on the fundamental cycle of $M$, is
integral, i.e. $c_{2}(P_{f})[M]\in \Z$. Thus, in this case, $P_{f}$ is classified by an integer $n(f)$ and we can write
$\mc{A}(P_{f})$ as $\mc{A}_{n(f)}$ or simply as $\mc{A}_{n}.$

In the physical literature, this number is referred to as the instanton number of $P_{f}$. From both the physical and
the mathematical point of view, this is the most important case.

We now restrict ourselves to a fixed member $\mc{A}(P_{f})$ of the disjoint union $A_{M}$ and write simply $P$ and
$P_{f}$. Then the \emph{Yang-Mills action} or \emph{functional} $\mc{A}_{YM}$ is defined by
\begin{equation}
 \label{MM:8.8}
 \mc{A}_{YM}(\om)=\frac{1}{8 \pi^{2}}\int_{M}\abs{F_{\om}}^{2}dv_{g}, \ \forall \om \in \mc{A}(P).
\end{equation}

To find the corresponding Euler-Lagrange equations, we take into account the fact that the space of gauge potentials is
an affine space and hence it is enough to consider variations along the straight lines through $\om$ of the form
$$
\om_{t}=\om+tA, \ A \in \Lam^{1}(M, \ad{} P).
$$

Direct computation shows that the gauge field, corresponding to the gauge potential $\om_{t}$, is given by
\begin{equation}
 \label{MM:8.9}
 F_{\om_{t}}=F_{\om}+td^{\om}A+t^{2}(A\wed A).
\end{equation}

Using equations (\ref{MM:8.8}) and (\ref{MM:8.9}) we obtain
\begin{align*}
 \frac{d}{dt}\mc{A}_{YM}(\om_{t})|_{t=0} &= \frac{d}{dt}\var{\int_{M}\abs{F_{\om}+td^{\om}A+t^{2}(A \wed
A)}_{x}^{2}dv_{g}}_{t=0} \\
&=2 \int_{M} \lan F_{\om}, d^{\om}A \ran dv_{g}\\
&=2 \int_{M} \lan \delta^{\om}, A \ran dv_{g}.
\end{align*}

The above result is often expressed in the form of a variational equation
\begin{equation}
 \label{8.10}
 \del \mc{A}_{YM}(\om)(A)=2\ave{\del^{\om}F_{\om}}{A}
\end{equation}

A gauge potential $\om$ is called a \emph{critical point} of the Yang-Mills functional if
\begin{equation}
 \label{MM:8.11}
 \del \ymf(\om)(A)=0, \ \forall A \in \Lam^{1}(M, \ad{} P).
\end{equation}

The critical points of the Yang-Mills functional are solutions of the corresponding Euler-Lagrange equations
\begin{equation}
 \label{MM:8.12}
 \del^{\om}F_{\om}=0.
\end{equation}

The equations (\ref{MM:8.12}) are called the pure (or sourceless) \emph{Yang-Mills equations}. A gauge potential $\om$
satisfying the Yang-Mills equations is called the \emph{Yang-Mills connection} and its gauge field $F_{\om}$ is called
the \emph{Yang-Mills fields}. Using local expressions for $d^{\om}$ and $\del^{\om}$ it can be shown that
$\del^{\om}=\pm \hdg d^{\om} \hdg.$ From this it follows that $\om$ satisfies the Yang-Mills equations (\ref{MM:8.12})
if and only if it is a solution of the equation
\begin{equation}
 \label{MM:8.13}
 d^{\om} \hdg F_{\om} =0.
\end{equation}

In the following theorem we have collected together various characterizations of the Yang-Mills equations.

\begin{thm}
 \label{MM:thm:8.2}
 Let $\om$ be a connection on the bundle $P(M,G)$ with finite Yang-Mills action. Then the following statements are
equivalent:
\begin{enumerate}[(i)]
 \item $\om$ is a critical point of the Yang-Mills functional;
 \item $\om$ satisfies the equation $\del^{\om}F_{\om}=0$;
 \item $\om$ satisfies the equation $d^{\om}\hdg F_{\om}=0$;
 \item $\om$ satisfies the equation $\nab^{\om}F_{\om}=0,$ where $\nab^{\om}=d^{\om}\del^{\om}+\del^{\om}d^{\om}$ is
the Hodge Laplacian on forms.
\end{enumerate}
\end{thm}

The equivalence of the first three statements follows form the discussion given above. The equivalence of the statement
(ii) and (iv) above follows from the identity
$$
\ave{\nab^{\om}F_{\om}}{F_{\om}}=\norm{d^{\om}F_{\om}}^{2}+\norm{\del^{\om}F_{\om}}^{2}=\norm{\del^{\om}F_{\om}}^{2}.
$$

The second equality in the above identity follows from the Bianchi identity
\begin{equation}
 \label{MM:8.14}
 d^{\om}F_{\om}=0.
\end{equation}

This identity is a consequence of the Cartan structure equations and express the fact that locally, $F_{\om}$ is
derived from a potential. It is customary to consider the pair (\ref{MM:8.12}) and (\ref{MM:8.14}) or (\ref{MM:8.13})
and (\ref{MM:8.14}) as the Yang-Mills equations. This is consistent with the fact that they reduce to Maxwell's
equations for the electromagnetic field $F$, when the gauge group $G$ is $U(1)$ and $M$ is the Minkowski space.

In local orthonormal coordinates, the Yang-Mills equations can be written as follows
\begin{equation}
 \label{MM:8.15}
 \frac{\p F_{ij}}{\p x^{i}}+\lie{A_{i}}{F_{ij}}=0,
 \end{equation}
where the components $F_{ij}$ of the 2-form $F_{\om}$ are given by
\begin{equation}
 \label{MM:8.16}
 F_{ij}=\frac{\p A_{j}}{\p x^{i}}-\frac{\p A_{i}}{\p x^{j}}+\lie{A_{i}}{A_{j}}.
\end{equation}

Thus we see that the Yang-Mills equations are a system of non-linear, second order, partial differential equations for
the components of the gauge potential $A$. The presence of quadratic and cubic terms in the potential is interpreted as
representing the self-interaction of the Yang-Mills field.

In (\ref{MM:8.8}) we have defined the Yang-Mills action $\ymf$ in the domain $\mc{A}(P).$ This domain is an affine
space and hence is contractible. But there is a large symmetry group acting on this space. This is the group $\mc{G}$
of gauge transformations of $P.$ By definition, a gauge transformation $\phi \in \mc{G}$ is an automorphism of the
bundle $P$, which covers the identity map of $M$ and hence leaves each $x\in M$ fixed. Under this gauge transformation,
the gauge field transforms as
$$
F_{\om} \mapsto F^{\phi}_{\om}:=\phi^{-1}F_{\om} \phi.
$$

Therefore, the pointwise norm $\abs{F_{\om}}$ is a gauge invariant. It follows that the Yang-Mills action $\ymf(\om)$
is gauge invariant and hence induces a functional on the moduli space $\mc{M}=\mc{A}/\mc{G}$ of gauge connections. This
functional is also called the Yang-Mills functional (or action). In order to relate the topology of the moduli space
$\mcm$ to the critical points of the Yang-Mills functional, it is necessary to compute the second variation or the
Hessian of the \ym action,
$$
\del^{2}\ymf(\om):=\frac{d^{2}}{dt^{2}}\ymf(\om_{t})|_{t=0}.
$$

One can verify that the Hessian, viewed as a symmetric, bilinear form, is given by the following expression:
$$
\del^{2}\ymf (\om )(A,B)=2\int_{M}\var{\inp{\dom A}{\dom B}+\inp{\fom}{A \wed B}},
$$

Analysis of the Yang-Mills equations then proceeds by obtaining various estimates using these variations.

We note that, if $\phi \in \aut{P}$ covers a conformal transformations of $M$ (also denoted by $\phi$), then thus
$\phi \in \diff{M}$ induces a conformal change of metric which is given by
\begin{equation}
 \label{MM:8.17}
 \phi^{*}g=e^{2f}g, \ f \in \mc{F}(M).
\end{equation}

Neither the pointwise norm $\abs{\fom}_{x}$ nor the Riemannian volume form $v_{g}$ are invariant under this conformal
change and the integrand in the \ym action transforms as follows:
\begin{equation}
 \label{MM:8.18}
 \abs{\fom}^{2}_{x}dv_{g}\mapsto \var{e^{-4f}\abs{\fom}^{2}_{x}}(e^{mf}dv_{g}).
\end{equation}

It follows that, in the particular case of $m=4$, the \ym action is invariant under the generalized gauge
transformations which cover conformal transformations of M.

A large number of solutions of the Yang-Mills equations are known for special manifolds. For example, the universal
connections on Stiefel bundles provide solutions of pure (sourceless) \ym equations. As we observed in the precious
section, the natural connections associated to the complex Hopf fibrations are particular cases of the Stiefel bundles.

Let $A$ be the local gauge potential corresponding to the universal connection $\om$ on the Stiefel bundle
$V_{k}(F^{n})(G_{k}(F^{n}), U_{F}(k))$ with gauge group $U_{F}(k)$ and let $\fom=\dom A$ be the corresponding gauge
field. The basic computation involves an explicit local expression for $\hdg \fom$, the Hodge dual of the gauge field
$F_{\om}$. In the complex case, it can be shown that $\hdg \fom$ is represented by an invariant polynomial of degree
$kl-1$, $l=n-k$ in $\fom$, i.e.
$$
\hdg F = a F^{\wed kl-1},
$$
where $a$ is a constant. Thus expression, together with the Bianchi identities, imply
\begin{equation}
 \label{MM:8.19}
 \dom \hdg \fom =0.
\end{equation}

In the quaternion case, a similar expression for $\hdg \fom$ leads to the equation (\ref{MM:8.19}). In the real case,
the expression for $\hdg F_{\om}$ also involves the gauge potential explicitly. These terms can be written as a wedge
product of certain $l$-forms $\psi$, which satisfy the conditions $d^{A}\psi=0$. As a consequence of the Mauer-Cartan
equation this, together with the Bianchi identities, implies (\ref{MM:8.19}) as in the complex case.

Let $M$ be a compact, oriented Riemannian manifold of dimension 4. Let $P(M, G)$ be a principal bundle over $M$ with
compact semisimple Lie group $G$ as  structure group. Recall that the Hodge star operator on $\Lam(M)$ has a natural
extension to the bundle valued forms. A form $\alp \in \Lam^{2}(M, \ad{}P)$ is said to be \emph{self-dual} (resp.
\emph{anti-dual}) if
$$
\hdg \alp = \alp \ (\text{resp. } \hdg \alp = - \alp).
$$

We define the \emph{self-dual part} $\alp_{+}$ of a form $\alp \in \Lam^{2}(M, \ad{} P)$ by
$$
\alp_{+}:= \frac{1}{2}(\alp + \hdg \alp).
$$
Similarly, the \emph{anti-dual part} $\alp_{-}$ of a form $\alp \in \df{2}{M, \ad{ }P}$ is defined by
$$
\alp_{-}:= \frac{1}{2}(\alp - \hdg \alp).
$$

Over a four dimensional base manifold, the second Chern class and the Euler class of P are equal and we define the
\emph{instanton number} $k$ of the bundle $P$ by
\begin{equation}
 \label{MM:8.20}
 k:= -c_{2}(P)[M]= - \chi(P)[M].
\end{equation}
recall that $F=\fom$ is the curvature form of the gauge connection $\om$, and hence, by the theory of characteristic
classes, we have
\begin{equation}
 \label{MM:8.21}
 k:=-c_{2}(P)[M] = -\frac{1}{8\pi^{2}} \int_{M} \Tr(F \wed F).
\end{equation}

Decomposing $F$ into its self-dual part $F_{+}$ and anti-dual part $F_{-}$, we get
\begin{equation}
\label{MM:8.22}
 k=\frac{1}{8\pi^{2}}\int_{M}\var{\abs{F_{+}}^{2}-\abs{F_{-}}^{2}}.
\end{equation}

Using $F_{+}$ and $F_{-}$ we can rewrite the \ym action \req{MM}{8.8} as follows:
\begin{equation}
 \label{MM:8.23}
 \ymf(\om)=\frac{1}{8\pi^{2}}\int_{M}\var{\abs{F_{+}}^{2}+\abs{F_{-}}^{2}}.
\end{equation}

Comparing equations \req{MM}{8.22} and \req{MM}{8.23} above we see that \ym action is bounded below by the absolute
value of the instanton number, i.e.
\begin{equation}
 \label{MM:8.24}
 \ymf(\om)\geq \abs{k}, \ \forall \om \in \mc{A}(P).
\end{equation}

When $M$ is four dimensional, we can associate to the pure \ym equations the first order \emph{instanton} (resp.
\emph{anti-instanton}) \emph{equations}
\begin{equation}
 \label{MM:8.25}
 \fom =\hdg \fom \ (\text{resp. } \fom=-\hdg \fom).
\end{equation}

The Bianchi identities imply that any solution of the instanton equations is also a solution of the \ym equations. The
fields satisfying $\fom= \hdg \fom$ (resp. $\fom = - \hdg \fom$) are also called \emph{self-dual} (resp.
\emph{anti-dual}) \ym fields. We note that a gauge connection satisfies the instanton or the anti-instanton equation if
and only if it is the absolute minimum of the \ym action. It can shown that the instanton equations are also invariant
under the generalized gauge transformations, which cover conformal transformations of $M$. This result plays a crucial
role in the construction of the moduli space of instantons.


We now give two examples of \ym fields on $S^{4}$ obtained by gluing two potentials along the equator.

\begin{exmp}
 We consider the two standard charts $U_{1}, U_{2}$ on $S^{4}$ and define the $SU\left( 2 \right)$ principal bundle
$P_{1}(S^{4}, SU(2))$ by the transition function
$$
g:U_{1} \cap U_{2} \to SU(2), \ x \mapsto x/\abs{x}
$$
where $x\in \R^{4}$ is identified with a quaternion. Essentially, the transition function $g$ defines a map of the
equator $S^{3}$ into $SU(2),$ i.e., an element of the homotopy group $[g] \in \pi_{3}(SU(2)) \cong \Z.$ In fact, $[g]$
is non-trivial, corresponding to the non-triviality of the bundle $P_{1}.$ A connection $A \in \mca (P_{1}(S^{4},
SU(2)))$ is specified by data consisting of a pair of $\mfk{su}(2)$-valued 1-forms $A^{i}$ on $U_{i}, \ i=1,2$, which,
when restricted to $U_{1}\cap U_{2}$, obey the cocycle condition
$$
A^{1}(x)=g(x)A^{2}g^{-1}(x)+g(x)dg^{-1}(x).
$$
For each $\lam \in (0,1)$ define the connection $A^{\lam}=(A^{\lam}_{1}, A^{\lam}_{2})$ by
$$
A^{\lam}_{1}= \op{Im}\var{\frac{xd\tilde{x}}{\lam^{2}+\abs{x}^{2}}}, \
A^{\lam}_{2}= \op{Im}\var{\frac{\lam^{2}\tilde{x}d{x}}{\abs{x}^{2}\var{\lam^{2}+\abs{x}^{2}}}}.
$$

The connection $A^{\lam}$ is self dual with instanton number 1. The curvature of this connection $A^{\lam}$ is
$F^{\lam+}=\var{F_{1}^{\lam +}, F_{2}^{\lam +}}$ given by
$$
F_{1}^{\lam +}= \op{Im}\var{\frac{\lam^{2}dx \wed d\tilde{x}}{(\lam^{2}+\abs{x}^{2})^{2}}},
F_{2}^{\lam +}= \op{Im}\var{\frac{\lam^{2}\tilde{x}dx \wed x d\tilde{x}}{\abs{x}^{2}(\lam^{2}+\abs{x}^{2})^{2}}}.
$$

The basic anti-instanton over $\R^{4}$ is described as follows. The principal bundle $P_{-1}(S^{4}, SU(2))$ is defined
by the transition function
$$
\bar{g}: U_{1} \cap U_{2}\to SU(2), \bar{g}=\bar{x}/\abs{x}.
$$

For each $\lam \in (0,1)$ define the connection $B^{\lam }=(B_{1}^{\lam}, B_{2}^{\lam})$ by
$$
A^{\lam}_{1}= \op{Im}\var{\frac{\tilde{x}dx}{\lam^{2}+\abs{x}^{2}}}, \
A^{\lam}_{2}= \op{Im}\var{\frac{\lam^{2}xd\tilde{x}}{\abs{x}^{2}\var{\lam^{2}+\abs{x}^{2}}}}.
$$
The connection $B^{\lam}$ is anti-dual with instanton number $-1$. We note that $[\bar{g}]=-[g]\in \pi_{3}(SU(2))\cong
\Z$ and thus $[\bar{g}]$ is also non-trivial. The curvature of this connection $B^{\lam}$ is
$F^{\lam-}=(F^{\lam-}_{1}, F_{2}^{\lam-})$ given by
$$
F_{1}^{\lam -}= \op{Im}\var{\frac{\lam^{2} d\tilde{x} \wed dx }{(\lam^{2}+\abs{x}^{2})^{2}}},
F_{2}^{\lam -}= \op{Im}\var{\frac{\lam^{2} x d\tilde{x} \wed \tilde{x}dx }{\abs{x}^{2}(\lam^{2}+\abs{x}^{2})^{2}}}.
$$
\end{exmp}



