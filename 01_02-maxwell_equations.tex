\section{Covariant form of Maxwell's equations}

Maxwell electromagnetic theory provides an important example of a gauge theory.  The Maxwell equations in a classical form are given by:

\begin{align}
\label{maxwell:gauss_mag}
\dvg B&=0\\
\label{maxwell:faraday}
\crl E +\p B / \p t &= 0 \\
\label{maxwell:gauss}
\dvg E &= \rho\\
\label{maxwell:ampere}
\crl B  - \p E / \p t  &= J
\end{align}

where the electric field $E$ and the magnetic field $B$ are time dependent vector fields on some subset of $\R^{3}$ and $\rho$ and $J$ are the charges and current densities respectively.

From (\ref{maxwell:gauss}) and (\ref{maxwell:ampere}), we obtain \ti{the continuty equation}

\begin{equation}
\label{maxwell:continuity_eq}
\dvg J + \pt{\rho}=0.
\end{equation}

From (\ref{maxwell:gauss_mag}), there is a vector function A, called the vector potential, such that
\begin{equation}
\label{SA:1.21}
B=\crl A.
\end{equation}

Subtituting into (\ref{maxwell:faraday}), we obtain
\begin{equation}
\label{SA:1.22}
\crl \var{E+\pt{A}}=0,
\end{equation}
and therefore, there is a scalar potential $\Phi$, such that
\begin{equation}
\label{SA:1.23}
E=-\nab \Phi - \pt{A}.
\end{equation}

Maxwell's equations can be written in \ti{covariant form} by introducing 
\begin{align}
\label{SA:1.24}
A^{\alpha}=&(\Phi, A)=(A^{0}, \dots, A^{3})\\
\label{SA:1.25}
J^{\alpha}=&(\rho, J)=(J^{0}, \dots, J^{3}).
\end{align}

Using this notation, we define 
\begin{equation}
\label{SA:1.28}
F^{\alpha \beta}=\p^{\alpha}A^{\beta}-\p^{\beta}A^{\alpha},
\end{equation}
and from (\ref{SA:1.21}) and (\ref{SA:1.22}), we obtain the \ti{contravariant field tensor}
\begin{equation}
\label{maxwell:contrav_tensor}
F^{\alpha \beta} =
\begin{bmatrix}
0 & E_{1} & E_{2} & E_{3} \\
-E_{1} & 0 & B_{3} & -B_{2} \\
-E_{2} & -B_{3} & 0 & B_{1} \\
-E_{3} & B_{2} & -B_{1} & 0
\end{bmatrix}.
\end{equation}

The components of the fields in equations (\ref{SA:1.21}) (\ref{SA:1.23}) can be identified as
\begin{align}
\label{SA:1.30}
E_{i}&=F^{0 i} \\
B_{i}&=\dfrac{1}{2}\epsilon^{i j k}F^{j k}, \ i,j,k=1,2,3,
\end{align}
where
$$
\epsilon^{i j k}=
\begin{cases}
1 & (i,j,k)= (1,2,3), (3,1,2), (2,3,1), \\
-1 & (i,j,k)=(1,3,2), (3,2,1), (2,1,3), \\
0 & otherwise
\end{cases}
$$
is the \ti{Levi-Civita symbol}.

For the \ti{covariant field tensor} defined by
\begin{equation}
F_{\alpha \beta}=\p_{\alpha}A_{\beta}-\p_{\beta}A_{\alpha},
\end{equation}
we obtain
\begin{equation}
\label{maxwell:cov_tensor}
F_{\alpha \beta} = \eta_{\alpha \gamma} \eta_{\beta \lambda} F^{\gamma \lambda}=
\begin{bmatrix}
0 & -E_{1} & -E_{2} & -E_{3} \\
E_{1} & 0 & B_{3} & -B_{2} \\
E_{2} & -B_{3} & 0 & B_{1} \\
E_{3} & B_{2} & -B_{1} & 0
\end{bmatrix}.
\end{equation}

The homogeneous Maxwell's equations (\ref{maxwell:gauss_mag}) and (\ref{maxwell:faraday}) correspond to the Jacobi identities:
\begin{equation}
\label{SA:1.35}
\pu{\gamma} F^{\alpha \beta}+ \pu{\alpha}F^{\beta \gamma} + \pu{\beta} F^{\gamma \alpha}=0,
\end{equation}
where $\alpha, \beta, \gamma \in \set{0,\dots,3}.$

For example, if $\gamma=1, \alpha=2, \beta=3$ we have from (\ref{maxwell:contrav_tensor}) and (\ref{SA:1.35}), 
\begin{equation}
\label{SA:1.36}
\pu{1}F^{32}+\pu{2}F^{13}+\pu{3}F^{21}= -(\pu{1}B_{1}+\dots +\pu{3}B_{3})
=-(\pd{1}B_{1}+\dots +\pd{3}B_{3})=0,
\end{equation}
which indeed corresponds to (\ref{maxwell:gauss_mag}).

The inhomogeneous Maxwell's equations (\ref{maxwell:ampere}) and (\ref{maxwell:gauss}) can be written as 
\begin{equation}
\label{SA:1.37}
\pd{\beta} F^{\alpha \beta}=J^{\alpha}.
\end{equation}

For example, if $\alpha=0,$ we have from (\ref{maxwell:contrav_tensor}) and (\ref{SA:1.37})
\begin{equation}
\label{SA:1.38}
\pd{0}F^{00}+\dots +\pd{3}F^{03}=\pd{1}E_{1}+\dots +\pd{3}E_{3}= \rho , 
\end{equation}
which indeed corresponds to (\ref{maxwell:ampere}).

\begin{rem}
Maxwell's equations have been reduced to (\ref{SA:1.35}) and (\ref{SA:1.37}). The continuity equation in covariant form can be obtained from (\ref{SA:1.37}) by operating $\p_{\mu}$ on both sides. Then 
\begin{equation}
\label{SA:1.39}
\pd{\alpha}J^{\alpha}=\pd{\alpha} \pd{\beta} F^{\alpha \beta}=0,
\end{equation} 
since $\pd{\alpha} \pd{\beta}$ is symmetric in $\alpha$ and $\beta$ while $F^{\alpha \beta}$ is antisimmetric in  $\alpha$ and $\beta.$ The expression (\ref{SA:1.39}) is \emph{the conservation of electric charge} whose underlying symmetry is gauge invariance.
\end{rem}