\section{The metric}

\begin{defn}
 \label{SA:3.1.1}
 A real vector space $V$ is called a metric vector space if on $ V $, a scalar product is defined as a map
 $$ 
 g:V \times V \to \R, 
  $$
such that for all $ u,v \in V $ and $ \Lambda \in \R $ the following propieties are satisfied:
\begin{description}
  \item[(Bilinear)] $g(\lambda u+v, w)= \lambda g(u,v)+ g(u,w) $
  \item[(Symmetric)] $g(u,v)=g(v,u)$
  \item[(Non-degenerate)] $\forall u \in V,g(u,v)=0 \Longleftrightarrow v=0$
\end{description}
If $e_{\alpha}$ is the orthonormal basis in $V,$ then
\begin{equation}
\label{SA:3.1}
g(e_{\alpha}, e_{\beta})=g_{\alpha \beta}=\pm\delta_{\alpha \beta}=
\begin{cases}
\pm 1 & \alpha=\beta \\
0 & \alpha\neq \beta
\end{cases}
\end{equation}
\end{defn}

The signature of the metric is defined by the number of $+1's$ and $-1's$, usually denoted by $(p,q).$ The idea if a
metric can be extended to the space $\mfk{F}(M)$ of all vector fields and the space $\Omega^{1}(M)$ of all 1-forms on a
manifold $M.$ On a smooth manifold $M,$ a metric $g$ assigns to each point $p \in M$ a metric $g_{p}$ on the tangent
space $T_{p}M$ in a smooth varying way
\[
g_{p}:T_{p}M \times T_{p}M \to \R,
\] 
which satisfies the above propieties with $\lambda$ replaced by $f \in C^{\infty}(M),$ and $g(u,v)$ is a funtion on $M$
whose value at $p$ is $g_{p}(u_{p},v_{p}),$ where $u,v \in \mfk{F}(M)$ and $u_{p},v_{p}\in T_{p}M.$

If the signature of $g$ is $(n,0),$ $n$ being the dimension of $M,$ we say that $g$ is a Riemmanian metric, while if $g$
has the signature of $(n-1,1),$ we say that $g$ is Lorentzian. A manifold equipped with a metric will be called a
semi-Riemmannian manifold denoted by $(M, g).$

Setting $\tilde{g}(u)(v)=g(u,v),$ we obtain an isomorphism
\begin{equation}
\label{SA:3.2}
\tilde{g}:T_{p}M \to T_{p}^{*}M,
\end{equation}
which can be proved by using the non-degeneracy propierty and the fact that $\dim T_{p}M= \dim T_{p}^{*}M. $ 

Let $\set{\pd{\alpha}}$ be a basis of a vector field on an open neighbourhood $U$ of a point in $M,$ then, the
components of the metric are given by
\begin{equation}
\label{SA:3.3}
g_{\alpha \beta}=g(\pd{\alpha}, \pd{\beta}).
\end{equation}

If the dimension of $M $ es $n$, then $g_{\alpha \beta}$ is an $n \times n$ matrix. The non-degeneracy property shows
that $g_{\alpha \beta}$ is invertible and we shall denote the inverse by $g^{\alpha \beta},$ also $g_{\alpha
\gamma}g^{\alpha \beta}=\delta_{\gamma}^{\beta}.$ This leads to the raising and lowering of indices which is a process
of converting vector fields to 1-forms. With the help of (\ref{SA:3.2}), one can easily convert between tangent vectors
and contangent vectors.

\begin{exmp}
\label{SA:3.1.2}
If $u=u^{\alpha}\pd{\alpha}$ is a vector field on a chart, then the corresponding 1-form can be calculated can be
calculated as follows:
\begin{align*}
 \tilde{g}(u)(v)&=g_{\alp \bet}u^{\alp}v^{\bet} \\
&=g_{\alp \gam}\delta^{\gam}_{\bet}u^{\alp}v^{\bet} \\
&=g_{\alp \gam}u^{\alp}v^{\bet}dx^{\gam}(\p^{\bet})\\
&=(g_{\alp \gam}u^{\alp}dx^{\gam})v^{\bet}\p_{\bet}= (u_{\gam}dx^{\gam})v
\end{align*}
Therefore $\tilde(u)=u_{\gam}dx^{\gam},$ where $u_{\gam}=g_{\alp \gam}u^{\alp}.$

Conversely, if $\xi=\xi_{\bet}dx^{\bet}$ is a 1-form on a chart, then the corresponding vector field can be calculated
as follows:
\begin{align*}
 \tilde{g}^{-1}(\chi)(\eta)&=g^{\bet \gam}\xi_{\bet}\eta_{\gam}\\
&=g^{\bet \alp}\xi_{\bet}\eta_{\gam}dx^{\gam}(\p_{\alp})\\
&= (\xi^{\alp}\p_{\alp})\eta,
\end{align*}
Therefore $\tilde{g}^{-1}(\xi)=\xi^{\alp}\p_{\alp},$ where $\xi^{\alp}=g^{\bet \alp}\xi_{\bet}.$
\end{exmp}

Using the fact that we can switch from 1-form to vector fields and vice versa with the help of a metric, we define the
inner product of the two 1-form $\xi, \eta$ as
\begin{equation}
 \label{SA:3.4}
\lan \xi, \eta \ran = g^{\bet \gam} \xi_{\bet} \eta_{\gam}.
\end{equation}

if $\tht^{1} \wed \cdots \wed \tht^{p}$ and $\sig^{1}\wed \cdots \wed \sig^{p}$ are orthonormal basis of $p-$forms,
then
\begin{equation}
 \label{SA:3.5}
\lan \tht^{1} \wed \cdots \wed \tht^{p}, \sig^{1}\wed \cdots \wed \sig^{p} \ran = \det \left[g(\tht^{\alp},
\sig^{\bet})\right],
\end{equation}
we define
\begin{equation}
 \label{SA:3.6}
\lan \tht^{\alp}, \tht^{\bet} \ran =
\begin{cases}
 \vep_{\bet} = \pm 1, & \alp=\bet \\
0, & \alp \neq \bet.
\end{cases}
\end{equation}

\begin{defn}
 \label{SA:3.1.3}
Let $M$ be an $n-$dimensional manifold, we define the volume form on a chart $(U_{\alp}, \phi_{\alp})$ as
$$
\mfk{Y}=dx^{1}\wed \cdots \wed dx^{n};
$$
if the manifold is a semi-Riemannian manifold we have,
$$
\mfk{Y}= \sqrt{\abs{\det g_{\alp \bet}}} dx^{1}\wed \cdots \wed dx^{n}.
$$
\end{defn}




