\section{The Vacuum Maxwell's Equations}

In free space or \emph{vacuum}, Maxwell's equations correspond to $\rho, \tb{J}=0,$ i.e., $J=0,$ which amounts to the
exchange 
$$
F \mapsto \hdg F.
$$

We say that $F \in \Om^{2}(M)$ is \emph{self-dual} if $\hdg F = F$ and \emph{anti-self-dual} if $\hdg F = - F.$ In
3-dimensional Riemannian manifold, it was shown that $\hdg^{2}=1.$ This implies that the Hodge star operator has
eigenvalues $\pm 1.$ Therefore, if we take $F_{\pm}=\frac{1}{2}(F\pm \hdg F),$ we can consider any $F \in \Om^{2}(M)$
as a sum of a self-dual and anti-self-dual:
$$
F=F_{+}+F_{-},
$$
where $\hdg F = \pm F_{\pm}.$

However, in the Lorentzian case $\hdg ^{2}=-1,$ which implies that the eigenvalues are $\pm i.$ If we consider
complex-valued differential forms on $M,$ it follows that, for any $F \in \Om^{2}(M),$ we have
$$
F=F_{+}+F_{-},
$$
where $\hdg F = \pm i F_{\pm}.$

In both cases, if $F$ is a self-dual or anti-self-dual 2-form satisfying 
\begin{equation}
 \label{SA:4.34}
dF=0,
\end{equation}
 automatically it satisfies 
\begin{equation}
 \label{SA:4.35}
\hdg d \hdg F = 0.
\end{equation}

Certainly, $F$ is a complex-valued in the Lorentzian case, but we can always split the real and the imaginary parts and
obtain a real solution using the fact that (vacuum) Maxwell's equations are linear, which correspond to either
(\ref{SA:4.34}) or (\ref{SA:4.35}).


