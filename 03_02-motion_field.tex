\section{Motion in an Electromagnetic Field}

We have discussed above the geometric setting which characterizes source-free electromagnetic fields. On the other
hand, the existence of an electromagnetic field has consequences for the geometry of the base space and this in turn
affects the motion of test particles.

First recall that an electrostatic field $E$ determines the differential of potential between two points by integration
along a path joining them. It is therefore reasonable to think of $E$ as a 1-form on $\R^{3}$. Maxwell's fundamental
idea was to introduce a quantity, called electric displacement, $D$ which has the propierty that its integral over a
closed surface measures the charge enclosed by the surface. One should thus think of $D$ as a 2-form. In a uniform
medium characterized by dielectric constant $\ep$, one usually writes the constitutive equations relating $D$ and $E$ as
$$
D=\ep E.
$$

However, our discussion indicates that this equation only makes sense if E is replaced by a 2-form. In fact, if we are
given a metric on $\R^{3},$ there is natural way to associate with $E$ a 2-form, namely $\hdg E,$ where $\hdg$ is the
Hodge operator. Then the above equation can be written in a mathematically precise form
$$
D=\ep(\hdg E).
$$

Conversely, requiring such a relation between $D$ and $E$, i.e., specifying the operator $\hdg: \Lam^{1}(\R^{3}) \to
\Lam^{2}(\R^{3})$, is equivalent to giving a Euclidean metric on $\R^{3}$. Similarly, ne can interprete \emph{the
magnetic tension} or \emph{induction} as a 2-form $B$ and Faraday's law then implies that $B$ is closed, i.e. $dB=0.$
The law of motion of a charged test particle of charge $e$, moving in the presence of $B$, is obtained by a
modification of the canonical symplectic form $\om$ on $T^{*}\R^{3}$ by the pull-back of $B$ by the canonical
projection $\pi: T^{*}\R^{3} \to \R^{3}$, i.e. by using the form 
$$
\om_{e,B}=\om + e\pi^{*}B.
$$

The separate theories of electricity and electromagnetism were unified by Maxwell in his electromagnetic theory. To
describe this theory for arbitrary fields, it is convenient to consider their formulation on the 4-dimensional
Minkowski space-time $M^{4}.$ We define the two 2-forms $F$ and $G$ by
$$
F:=B+E \wed dt, \ G:=D-H\wed dt,
$$
where $B$ and $H$ are the magnetic tension and magnetic field respectively and $E$ and $D$ are the electric field and
displacement respectively. If $J$ denotes the current 3-form, then Maxwell's equations with source $J$ are given by
\begin{align*}
 dF&=0\\
dG & = 4 \pi J.
\end{align*}
 
The two fieldsn $F$ and $G$ are related by the constitutive equations which depend on the medium in which they are
defined. In a uniform medium (in partiuclar, in vacuum), we can write the constitutive equations as 
$$
G=(\ep/\mu)^{1/2}\hdg F.
$$

If we choose the standard Minkowski metric, then we can write the source-free Maxwell's equations in a uniform medium as
\begin{align*}
 dF&=0 \\
d \hdg F &= 0.
\end{align*}
The second equation is equivalent to $\delta F =0;$ we thus obtain the gauge field equations discussed earlier.


It is well know that Hamilton's equation of motion of a particle in classical mechanics can be given a geometrical
formulation by using the phase space $P$ of the particle. The phase space $P$ is, at least locally, the cotangent
space $T^{*}Q$ of the configuration space $Q$ of the particle. We now show that this formalism can be extended to the
motion of a charged particle in an electromagnetic field and leads to the equations of motions with the Lorentz
force.We choose the configuration space $Q$ of the particle as the usual Minkowski space. The law of motion of a
charged test particle with charge $e$ is obtained by considering the symplectic form
$$
\om_{e,F}=\om + e \pi^{*}F,
$$
where $\om$ is the canonical symplectic form of $T^{*}Q$ and $\pi:T^{*Q} \to Q$ is the canonical projection.

The Hamiltonian is given by
$$
\Ham(q,p)=\frac{1}{2m}g^{ij}p_{i}p_{j},
$$
where $g_{ij}$ are the components of the Lorentz metric and $p_{i}$ are the components of 4-momentum. In the usual
system of coordinates we can write the metric and the Hamiltonian as
\begin{align*}
 ds^{2}=& dq_{o}^{2}-dq_{1}^{2}-dq_{2}^{2}-dq_{3}^{2},\\
\Ham(q,p)=\frac{1}{2m}(p_{o}^{2}-p_{1}^{2}-p_{2}^{2}-p_{3}^{2}).
\end{align*}
The matric of the symplectic form $\om_{e, F},$ in local coordinates $x^{\alp}=(q^{i}, p_{i})$ on $T^{*}Q$, can be
written in block matrix form as
$$
\om_{e,F} =
\begin{pmatrix}
 eF & -I \\
I & 0
\end{pmatrix}.
$$

The Hamiltonian vector field is given by
$$
X_{\Ham}=(d\Ham)^{\sharp}=\frac{\p \Ham}{\p x^{\alp}} \om^{\alp \bet}_{e, F} \p_{\bet},
$$
where $\om^{\alp \bet}_{e, F}$ are the elements of $(\om_{e, F})^{-1}.$ The corresponding Hamilton's equations are
given by
$$
\frac{dx^{\alp}}{dt}=X^{\alp}_{\Ham}=\om\frac{\alp \bet}{e, F}\frac{\p \Ham}{\p x^{\bet}}
$$
i.e.
\begin{align}
 \label{MM:8.4}
\frac{dq^{i}}{dt}&=\frac{\p \Ham}{\p p_{i}} \\
\label{MM:8.5}
\frac{dp_{i}}{dt}&=-\frac{\p \Ham}{\p q^{i}}+eF_{ij}\frac{\p \Ham}{\p p_{i}}. 
\end{align}

Equation (\ref{MM:8.5}) implies that the 3-momentum $p=(p_{1}, p_{2}, p_{3})$ of the particle satisfies the equation
\begin{equation}
 \label{MM:8.6}
\frac{dp}{dt}=\frac{e}{2m}(E + p \times B),
\end{equation}
where $E$ and $B$ are respectively the electric and magnetic fields. The equation (\ref{MM:8.6}) is the equation of
motion of a charged particle in a electromagnetic field subject to the Lorentz force. Now electromagnetic field is a
gauge field with abelian group $U(1).$

The orbits of contragredient action of $U(1)$ on the dual of its Lie algebra  $\mfk{u}(1)$ are trivial. Identifying
$\mfk{u}(1)$ and its dual with $\R,$ we see that an orbit through $e\in \R$ is the point $e$ itself. Thus, in this
case, the choice of an orbit is the same thing as the choice of the unit of charge. This construction can be
generalized to gauge fields with arbitrary structure group and, in particular, to the Yang-Mills fields to obtain the
equations of motion of a particle moving in a Yang-Mills fields. 
