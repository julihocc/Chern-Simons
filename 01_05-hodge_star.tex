\section{The Hodge Star Operator}

\begin{defn}
 \label{SA:3.2.1}
The Hodge star operator is the unique linear map on a semi-Riemannian manifold from $p-$forms to $(n-p)-$forms defined
by
$$
\hod: \Om^{p}(M) \to \Om^{(n-p)(M)},
$$
such that for all $\xi, \eta \in \Om^{p}(M)$,

$$\xi \wed \hod \eta =\inp{\xi}{\eta}\mfk{Y}.$$
\end{defn}


This is an isomorphism between $p-$forms and $(n-p)-$forms, and $\hod \eta$ is called the dual of $\eta.$ Suppose that
$dx^{1}, \cdots, dx^{n}$ are positively oriented orthonormal basis of 1-forms on some chart $(U_{\alp}, \phi_{\alp})$
on a manifold M.

Let $1 \leq i_{1} < \dots < i_{p} \leq n$ be an ordered distinct increasing indices and let $j_{1}< \dots <j_{n-p}$ be
their complement in the set $\set{1,...,n}.$ Then
\begin{equation}
\label{SA:3.8}
 dx^{i_{1}} \wed \cdots \wed dx^{i_{p}}\wed dx^{j_{1}}\wed \cdots \wed dx^{j_{n-p}}= \sgn(I)dx^{1}\wed \cdots \wed
dx^{n},
\end{equation}
where, $\sgn(I)$ es the sign of the permutation
$$
\begin{pmatrix}
i_{1} & ... & i_{p} & j_{1} & ... & j_{n-p} \\
1 & ... & p & p+1 & ... & n
\end{pmatrix}.
$$
 In other words, the wedge products of a $p-$forms and $(n-p)-$forms yield the volume form up to a sign.

\begin{rem}
 $\hod 1 \in \Om^{n}(M)$ is the volume form $\mfk{Y}$. For a domain $D\subset M,$ $\int_{D} \mfk{Y}$ is the volume of
$D.$ In particular, if $M$ is compact, $\int_{M}\mfk{Y}$ is called the volume of $M.$
\end{rem}

We state the following proposition without proof:

\begin{prop}[\cite{SM}, 4.7]
\label{SM:4.7}
 The $\hod-$ operator of Hodge has the following properties. For any $f, g \in C^{\infty}(M)$ and $\xi, \eta \in
\Om^{p}(M)$ we have
\begin{enumerate}[(i)]
 \item $\hod(f\xi + g \eta)= f \hod \xi + g \hod \eta$
\item $\hdg \hdg \xi = (-1)^{p(n-p)} \xi$
\item $\xi \wed \hdg \eta= \eta \wed \hdg \xi$
\item $\hdg (\xi \wed \hdg \eta)= \inp{\xi}{\eta}$
\item $\inp{\hdg \xi}{\hdg \eta}=\inp{\xi}{\eta}$
\end{enumerate}
\end{prop}

Now let $M$ be an oriented Riemannian manifold. For any $X \in \mfk{F}(M),$ let $\xi_{X}$ be the 1-form corresponding
to $X$ by the isomorphism $\mfk{F}(M) \cong \Om^{1}(M).$ We set
$$
\dvg X = \hdg d \hdg \xi_{X}.
$$
It is called the \emph{divergence} of $X$. If $M$ comes with opposite orientation, $\hdg$ changes to $-\hdg.$ Since
$\hdg$ appears twice in the definition of $\dvg,$ it follows that $\dvg X$ is defined independently of the coise of
orientation.

\emph{Let $I_{n}$ denote the set $\set{1,...,n}.$}

\begin{exmp}
 \label{SM:4.8}
On the Euclidean space $\R^{n},$ to $X=\sum_{k \in I_{n}}u^{k}\p_{k},$ there correspond
$\xi_{X}=\sum_{k \in I_{n}}u^{k}dx^{k}.$ Hence we
get
\begin{equation}
\label{SA:3.28}
 \dvg X = \sum_{k \in I_{n}} \p_{k}u^{k}
\end{equation}

by direct computation. This is the original definition of divergence.
\end{exmp}

We state the following theorem without proof:

\begin{thm}[\cite{SM}, 4.9]
 Let $M$ be an oriented compact Riemannian manifold. If $X$ is a vector field on $M,$ then we have the equality
$$
\int_{M} \dvg X \ \mfk{Y}_{M} = \int_{\p M} \inp{X}{n} \ \mfk{Y}_{\p M},
$$
where $n$ is the outward unit normal vector field on $\p M.$ In particular, if $M$ is a closed manifold, then
$$
\int_{M} \dvg X \ \mfk{Y}_{M} = 0.
$$
\end{thm}

\begin{rem}
 Let $d:\Om^{\star}(\R^{3}) \to \Om^{\star}(\R^{3})$ be the exterior derivative, $f\in C^{\infty}(M)$ and $A\in
\Om^{1}(\R^{3})$. Then
\begin{equation}
 \label{SA:3.24}
df=\sumi{k} \p_{k}f dx^{k}.
\end{equation}

Thus, for $dx=(dx^{1}, dx^{2}, dx^{3}),$
\begin{equation}
 \label{SA:3.25}
df=\inp{\nab f}{dx}.
\end{equation}

Let $A=\sumi{k}A_{k}dx^{k}$ be a 1-form in $\R^{3}.$ Then
\begin{equation}
 \label{SA:3.26}
\begin{split}
 dA &= \sumi{n} (\p_{\sn{1}}A_{\sn{2}}-\p_{\sn{2}}A_{\sn{1}})dx^{\sn{1}}\wed dx^{\sn{2}}, \\
\hdg d A &= \sumi{n} (\p_{\sn{2}}A_{\sn{3}}-\p_{\sn{3}}A_{\sn{2}})dx^{\sn{1}},
\end{split}
\end{equation}
where $\sn{\cdot}=(1 \ 2 \ 3)^{n}\in S_{3}.$ Therefore,
\begin{equation}
 \label{SA:3.27}
\hdg d A=\inp{(\crl A)}{dx}.
\end{equation}

In this way, $d(df)= \crl(\nab f)=0$ and $d(dA)=\dvg(\crl A)=0.$
\end{rem}

